\documentclass[a4paper]{article}
\usepackage[utf8]{inputenc} % Skal passe til editorens indstillinger
\usepackage[english]{babel} % danske overskrifter

\newcommand{\name}{Søren Krogh Andersen,\quad Carsten Nielsen}
\newcommand{\stnumber}{s123369,\quad s123161}
\newcommand{\course}{02155 Computer architecture}
\newcommand{\university}{Tehnical University of Denmark}
\newcommand{\studyline}{DTU Elektrotechnology}
\newcommand{\assignment}{Home Assignment 2}
\renewcommand{\date}{\today} %If another date, than that of today is desiered


% Palatino for rm and math | Helvetica for ss | Courier for tt
\usepackage{mathpazo} % math & rm
\linespread{1.05}        % Palatino needs more leading (space between lines)
\usepackage{palatino} % tt
\normalfont
\usepackage[T1]{fontenc}

\usepackage{graphicx}%allerese hentet % indsættelse af billeder
\usepackage{epstopdf} %Tilfj "--enable-write18" i argumentet for LaTex build. Dette vil konvertere .eps figurer til pdf-format
\graphicspath{{./picture/}} % stivej til bibliotek med figurer
\usepackage{subcaption} %Til gruppering af figurer
\usepackage{amsmath} %matpakke
\usepackage{amsfonts} %
\usepackage{amssymb} %
\usepackage{polynom} %for displaying polynom division
\usepackage{mathtools} % matematik - understøtter muligheden for at bruge \eqref{}
\usepackage{float}
\usepackage{hhline}
\usepackage[usenames,dvipsnames]{xcolor} %must be before loading other packages that use xcolor 
\usepackage{tikz}
\usepackage{pgfplots}
\usepackage{pgfplotstable}

%
\usepackage[compact,explicit]{titlesec}% http://ctan.org/pkg/titlesec
%

%Listings%
\usepackage{listingsutf8}
\usepackage{packs/mips}
%setup listings
\lstset{ %
  language=[mips]Assembler,       % the language of the code
  basicstyle=\footnotesize,       % the size of the fonts that are used for the code
  numbers=left,                   % where to put the line-numbers
  numberstyle=\tiny\color{gray},  % the style that is used for the line-numbers
  stepnumber=1,                   % the step between two line-numbers. If it's 1, each line 
                                  % will be numbered
  numbersep=5pt,                  % how far the line-numbers are from the code
  backgroundcolor=\color{white},  % choose the background color. You must add \usepackage{color}
  showspaces=false,               % show spaces adding particular underscores
  showstringspaces=false,         % underline spaces within strings
  showtabs=false,                 % show tabs within strings adding particular underscores
  frame=single,                   % adds a frame around the code
  rulecolor=\color{black},        % if not set, the frame-color may be changed on line-breaks within not-black text (e.g. commens (green here))
  tabsize=4,                      % sets default tabsize to 2 spaces
  captionpos=b,                   % sets the caption-position to bottom
  breaklines=true,                % sets automatic line breaking
  breakatwhitespace=false,        % sets if automatic breaks should only happen at whitespace
  title=\lstname,                 % show the filename of files included with \lstinputlisting;
                                  % also try caption instead of title
  keywordstyle=\color{blue},          % keyword style
  commentstyle=\color{Plum},       % comment style
  stringstyle=\color{mauve},         % string literal style
  escapeinside={\%*}{*)},            % if you want to add a comment within your code
  morekeywords={*,...}               % if you want to add more keywords to the set
}
 %Listings slut%









%"Matematik'' setioner
%\renewcommand\thesection{Question~\arabic{section}} %pas p�, kun i matematik
%\renewcommand\thesubsection{\thesection,~\alph{subsection}}
%\definecolor{MagRed}{RGB}{190,40,15}

%\titleformat{\section}{\normalfont\sffamily\large\bfseries\color{MagRed}}{}{0pt}{|\kern-0.15ex|\kern-0.15ex|\kern-0.15ex|~Question~\arabic{section}\qquad\quad#1\label{\arabic{section}}}
%\titleformat{\subsection}[runin]{\large\bfseries}{}{10pt}{\alph{subsection})~#1\label{\arabic{section}\alph{subsection}}}
%\titleformat{\subsubsection}[runin]{\itshape}{}{0pt}{~#1\label{\arabic{section}\alph{subsection}\arabic{subsubsection}}}
%\titleformat{\subsubsection}{\bfseries}{}{0pt}{\alph{subsection}.\arabic{subsubsection})\qquad\quad#1\label{\arabic{section}\alph{subsection}\arabic{subsubsection}}}



%Matematik hurtige ting
%fed
\renewcommand\vec[1]{\mathbf{#1}}
\newcommand\matr[3]{{}_{#2}\mathbf{#1}{}_{#3}}
\newcommand\facit[1]{\underline{\underline{#1}}}
%\renewcommand\d[3]{\frac{\mbox{d}^{#3}#1(#2)}{\mbox{d}#2^{#3}}}
%underline
%\renewcommand\vec[1]{\underline{#1}}
%\newcommand\matr[3]{{}_{#2}\underline{\underline{#1}}{}_{#3}}

\renewcommand\matrix[4]{ %{alignment}{to space}{from space}{matrix}
{\vphantom{\left[\begin{array}{#1}#4\end{array}\right]}}_{#2}\kern-0.5ex
\left[\begin{array}{#1}
#4
\end{array}\right]_{#3}
}
\newcommand\e[0]{\mbox{e}}
\newcommand\im[0]{i}

\newcommand\Jaco{\mbox{Jacobi}}
\newcommand\del[2]{\frac{\partial {#1}}{\partial {#2}}}
\newcommand\abs[1]{\left| {#1} \right|}
\newcommand\stdfig[4]{ %width,img,cap,lab
\begin{figure}[hb!]
\centering
\includegraphics[width={#1}\textwidth]{#2}
\caption{#3} \label{#4}
\end{figure}
}
\newcommand\diff{\dot}
\newcommand\ddiff{\ddot}


\textwidth = 400pt
\marginparwidth = 86pt
\hoffset = -25pt
\voffset= -10pt
\textheight = 630pt
\newcommand{\HRule}{\rule{\linewidth}{0.5mm}}
\usepackage{fancyhdr}
\usepackage[plainpages=false,pdfpagelabels,pageanchor=false]{hyperref} % aktive links
\hypersetup{%
  pdfauthor={\name - \stnumber},
  pdftitle={\assignment},
  pdfsubject={\course}}
%\usepackage{memhfixc}% rettelser til hyperref


\fancyhf{} % tom header/footer
\fancyhfoffset{20pt}
\fancyhfoffset{20pt}
\fancyhead[OL]{\name \\ \course}
\fancyhead[OC]{Date \\ \date}
\fancyhead[OR]{\university\\ \studyline}
\fancyfoot[FL]{}
\fancyfoot[FC]{\thepage}
\fancyfoot[FR]{}
\renewcommand{\headrulewidth}{0.4pt}
\renewcommand{\footrulewidth}{0.4pt}
\pagestyle{fancy}

% How to make ref to books or urls in bib
%\citetitle[fx: page 1]{name of ref in bib}
