\section{Handling variable record sizes and optimizing bubblesort and quicksort}
One apparent problem with the algorithms used in the previous section is the amount of needless swapping taking place, more so
for bubblesort than quicksort. To solve this problem we have modified the algorithms to start by creating a new array, copying 
the record keys and their index in the original array. Then this array is sorted according to keys and the index is used to copy 
records from the original array and then restore them in the original array in the correct positions. If the sorted data does not have
to be in the same memory location after the sorting, it is possible to reduce the runtime significantly by simply pointing to the newly
created array. However, we assume that the data must stay sorted in the original location.


The modified code can also handle records of variable length, defined at
runtime, length, and width is suppied as an argument to the function call, 
for simplicity in the test constans defined by \emph{\#define}s are used for
simplicity.
% The modified code can also handle records of variable length, defined at runtime, but the variable length is managed by \#defines
% in the test code for simplicity.
Our test code is shown in listing~\ref{varlist}. 
\subsection{Test results}
The test of the variable record length versions of quicksort and bubblesort was performed in the same way as the fixed record length ones.
\subsubsection{Bubblesort}
The test results for the variable record length bubblesort can be seen in figure (~\ref{fig:bubbleVarResults})
\begin{figure}[H]
	\flushleft
	\begin{minipage}[c]{0.4\textwidth}
		\scalebox{0.5}{
			\begin{tabular}{c | r | r | r }
				Block size / words & Blocks per set & hit \% & Execution time / ns \\ \hline
				1& 1& 55& 29344780 \\ \hline
				1& 2& 57& 28348585 \\ \hline
				1& 4& 57& 27946935 \\ \hline  
				2& 1& 58& 29083395 \\ \hline
				2& 2& 59& 28045527 \\ \hline
				2& 4& 59& 27629527 \\ \hline
				4& 1& 78& 17146977 \\ \hline
				4& 2& 78& 16626613 \\ \hline
				4& 4& 79& 16419809 \\ \hline
				8& 1& 89& 11638889 \\ \hline
				8& 2& 89& 11354073 \\ \hline
				8& 4& 90& 11239161 \\ \hline
				16&1& 95& 8873675 \\ \hline
				16&2& 95& 8706443 \\ \hline
				16&4& 95& 8643555  \\ \hline
				32&1& 97& 7988865 \\ \hline
				32&2& 97& 7842091  \\ \hline
				32&4& 97& 7791009 \\ \hline
				64&1& 99& 7140151\\ \hline
				64&2& 99& 7005623 \\ \hline
				64&4& 99& 6953655 \\ \hline
		\end{tabular} }
	\end{minipage}
	\begin{minipage}[c]{0.4\textwidth}
		\begin{tikzpicture}
			\begin{loglogaxis}[log basis x=2, log basis y=2, xlabel=Block size / words, ylabel=Blocks per set, zlabel=execution time / ns,
					xtick={1,2,4,8,16,32,64},
					ytick={1,2,4},
					view = {120}{35},% important to draw x,y in increasing order
					xmin = 0,
					ymin = 0,
					xmax = 128,
					ymax = 8,
					zmin = 0,
					unbounded coords = jump,
					colormap={pos}{color(0cm)=(white); color(6cm)=(orange)}
				]
				\addplot3[surf,mark=none, shader=flat, draw=mapped color!80!black] file {source/python/bubblesort_n.dat};

			\end{loglogaxis}
		\end{tikzpicture}
	\end{minipage}
	\caption{Test results for variable record length bubblesort with a cache size of 512 words.}
	\label{fig:bubbleVarResults}
\end{figure}
From the test results we can conclude that the modified bubblesort is much faster than the original by approximately a factor of 10.
The hit rate is much lower than the original bubblesort for small block sizes. This is a result of the large amount of copying taking
place from random locations, when the algorithm is about to complete.
The actual sorting part of the algorithm has a hit rate in the 90th percentile. It 
can also be seen that there is no significant difference in execution time for block sizes 1 and 2. This is because the array being
sorted always accesses memory with a stride of 2 when comparing. Thus a miss will always happen on comparison for these block sizes.
Furthermore we note that the associativity level does not affect the execution time like it did for the original bubblesort. 
This is due to the nature of bubblesort. Since the algorithm always compares elements that are next to each other, eviction of 
elements that will be needed in the near future is very rare. The original bubblesort showed a performance gain from increased 
associativity for large block sizes which can be explained by the higher probability that an element in the record would cross a
block boundary and thus initiate an entire block transfer, since the entire record was swapped.

\subsubsection{Quicksort}
The quicksort test was performed on the [1000][7] records array and the results can be seen in figure (~\ref{fig:quickVarResults})
\begin{figure}[H]
	\flushleft
	\begin{minipage}[c]{0.4\textwidth}
		\scalebox{0.5}{
			\begin{tabular}{c | r | r | r }
				Block size / words & Blocks per set & hit \% & Execution time / ns \\ \hline
				 1& 1& 76&   7007629 \\ \hline
				 1& 2& 76&	 7012779 \\ \hline
				 1& 4& 76&   7015979 \\ \hline
				 2& 1& 86&   4149107 \\ \hline
				 2& 2& 86&   4139955 \\ \hline
				 2& 4& 86&   4143023 \\ \hline
				 4& 1& 92&   2495663 \\ \hline
				 4& 2& 93&   2560511 \\ \hline
				 4& 4& 93&   2461655 \\ \hline
				 8& 1& 95&   1784063 \\ \hline
				 8& 2& 96&   1696759 \\ \hline
				 8& 4& 96&   1695359 \\ \hline
				 16&1& 96&   1509371 \\ \hline
				 16&2& 98&   1299963 \\ \hline
				 16&4& 98&   1295611 \\ \hline
				 32&1& 95&   1858767 \\ \hline
				 32&2& 98&   1214799 \\ \hline
				 32&4& 98&   1206255 \\ \hline
				 64&1& 92&   2867491 \\ \hline
				 64&2& 99&   1165347 \\ \hline
				 64&4& 99&   1134499 \\ \hline
		\end{tabular} }
	\end{minipage}
	\begin{minipage}[c]{0.4\textwidth}
		\begin{tikzpicture}
			\begin{loglogaxis}[log basis x=2, log basis y=2, xlabel=Block size / words, ylabel=Blocks per set, zlabel=execution time / ns,
					xtick={1,2,4,8,16,32,64},
					ytick={1,2,4},
					view = {120}{35},% important to draw x,y in increasing order
					xmin = 0,
					ymin = 0,
					xmax = 128,
					ymax = 8,
					zmin = 0,
					unbounded coords = jump,
					colormap={pos}{color(0cm)=(white); color(6cm)=(orange)}
				]
				\addplot3[surf,mark=none, shader=flat, draw=mapped color!80!black] file {source/python/quicksort_n.dat};

			\end{loglogaxis}
		\end{tikzpicture}
	\end{minipage}
	\caption{Test results for variable record length quick with a cache size of 512 words.}
	\label{fig:quickVarResults}
\end{figure}
Interestingly the modified quicksort executes slower than the original quicksort and has the same relative dependency on
cache parameters. It is clear that the overhead associated with modified algorithm is not beneficial for inputs of this size. 
The execution time will however improve compared to the normal quicksort for larger input sizes. 
For record sizes of 3-64 we have not measured an improvement, but the execution time of the modified algorithm approaches the original.
Therefore the modified algorithm is inefficient for the input sizes given in the exercise manual.
