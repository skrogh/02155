\section{Bubblesort and quicksort}

\subsection{Bubblesort}
The bubblesort algorithm given in the exercise manual contains two bugs. It will not sort the first element of a given list and the 
outer loop goes out of bounds. The algorithm also needs to be modified to sort both the record keys and values. 
The fixed code is shown below together with the main function that executes the test.
\lstinputlisting[language=C, label=bubble1list,caption=Bubblesort test program]{source/c/bubblesort.c}
Since arrays in C are organized in row-major order, it is expected that the bubblesort algoritm will benefit greatly from increased
block size in the cache. This is due to the fact that bubblesort, for a record length of seven, compares elements in memory with a 
stride of 7, and accesses all elements between the first element accessed and \(2*RECORD\_LENGTH\) on a swap. Thus, for the naive
bubblesort, increasing block size should always improve performance. The bubblesort algorithm shown in the listing above, however, 
exploits the fact that the largest element will always be correctly placed after the first iteration, the second largest element
after the second iteration and so forth. This places an upper bound on the effectiveness of increasing block size as unused words
will be needlessly need to be loaded and replaced in the cache. If associativity is increased, this upper bound should move back
further since large blocks will be able to coexist in the same set. Thus removing the penalty for accessing only a few 
elements, that map to the same block, outside of a large block.

\subsubsection{Bubblesort test}
Testing bubblesorts dependency on cache parameters was done by running the bubblesort test program for all power of two combinationes
of block sizes 1-64 words and 1-4 blocks per set, as associativity greater than 4 is impractical [~\ref{HandP}]. The total cache size
was kept constant at 512 words.

\begin{figure}[H]
	\flushleft
	\begin{minipage}[c]{0.4\textwidth}
		\scalebox{0.5}{
			\begin{tabular}{c | r | r | r }
				Block size / words & Blocks per set & hit \% & Execution time / ns \\ \hline
				1& 1& 78& 425391272 \\ \hline
				1& 2& 78& 413453090 \\ \hline
				1& 4& 78& 412887614 \\ \hline  
				2& 1& 88& 235780820 \\ \hline
				2& 2& 88& 232451104 \\ \hline
				2& 4& 88& 231967648 \\ \hline
				4& 1& 92& 140191677 \\ \hline
				4& 2& 92& 134672553 \\ \hline
				4& 4& 93& 134200693 \\ \hline
				8& 1& 94&  99320731 \\ \hline
				8& 2& 96&  88988171 \\ \hline
				8& 4& 96&  88561691 \\ \hline
				16&1& 95&  81937382 \\ \hline
				16&2& 98&  63179822 \\ \hline
				16&4& 98&  62679254  \\ \hline
				32&1& 94& 103627999 \\ \hline
				32&2& 99&  54189887  \\ \hline
				32&4& 99&  53796655 \\ \hline
				64&1& 91& 167612952 \\ \hline
				64&2& 99&  45967056 \\ \hline
				64&4& 99&  45382728 \\ \hline
		\end{tabular} }
	\end{minipage}
	\begin{minipage}[c]{0.4\textwidth}
		\begin{tikzpicture}
			\begin{loglogaxis}[log basis x=2, log basis y=2, xlabel=Block size / words, ylabel=Blocks per set, zlabel=execution time / ns,
					xtick={1,2,4,8,16,32,64},
					ytick={1,2,4},
					view = {120}{35},% important to draw x,y in increasing order
					xmin = 0,
					ymin = 0,
					xmax = 128,
					ymax = 8,
					zmin = 0,
					unbounded coords = jump,
					colormap={pos}{color(0cm)=(white); color(6cm)=(orange)}
				]
				\addplot3[surf,mark=none, shader=flat, draw=mapped color!80!black] file {source/python/bubblesort.dat};

			\end{loglogaxis}
		\end{tikzpicture}
	\end{minipage}
	\caption{Test results for bubblesort with a cache size of 512 words.}
	\label{fig:bubbleResults}
\end{figure}
The test results can be seen in figure (~\ref{fig:bubbleResults}).

From the results we can see that the maximum blocksize for a directmapped cache, before execution time starts rising again, is 
16 words. This is just as expected, as the record size the algorithm is working on contains records of size 7.
Thus there is a high probability that a swap sequence will have all the elements to be swapped already in cache,
because the preceding compare action already brought them in.

\subsection{Quicksort}
The quicksort algorithm given in the exercise manual does not contain any bugs, but needs to be modified to handle records instead
of single integers. The modified algorithm is shown below.
\lstinputlisting[language=C,label=quicksort1list, caption=Quicksort test program]{source/c/quicksort.c}
For quicksort we do not expect the same dependency on cache parameters as for bubblesort. This is due to quicksort not taking
having the same spatial locality as bubblesort, since quicksort compares elements on opposite sides of a certain point. 
The recording sorting quicksort will however benefit from increased block size as less cache misses will occur when swapping if
more elements of a single record are brought into cache when attempting to access the record key for comparison. Associativity is
not expected to greatly affect the execution time unless the block size is large. This is again due to the loading/eviction of 
unnecessary elements.

\subsubsection{Quicksort test}
Quicksort was tested for the same cache parameters as bubblesort. The results can be seen in figure (~\ref{fig:quickResults}).
\begin{figure}[H]
	\flushleft
	\begin{minipage}[c]{0.4\textwidth}
		\scalebox{0.5}{
			\begin{tabular}{c | r | r | r }
				Block size / words & Blocks per set & hit \% & Execution time / ns \\ \hline
				 1& 1& 74&   3431095 \\ \hline
				 1& 2& 74&	 3451395 \\ \hline
				 1& 4& 73&   3478195 \\ \hline
				 2& 1& 84&   2075397 \\ \hline
				 2& 2& 84&   2077737 \\ \hline
				 2& 4& 94&   2090477 \\ \hline
				 4& 1& 88&   1361177 \\ \hline
				 4& 2& 89&   1354573 \\ \hline
				 4& 4& 88&   1358421 \\ \hline
				 8& 1& 92&   1004953 \\ \hline
				 8& 2& 92&    994313 \\ \hline
				 8& 4& 92&    997337 \\ \hline
				 16&1& 95&    752365 \\ \hline
				 16&2& 96&    709037 \\ \hline
				 16&4& 96&    708525 \\ \hline
				 32&1& 96&    753393 \\ \hline
				 32&2& 98&    623945 \\ \hline
				 32&4& 98&    622065 \\ \hline
				 64&1& 95&    937653 \\ \hline
				 64&2& 99&    545204 \\ \hline
				 64&4& 99&    537269 \\ \hline
		\end{tabular} }
	\end{minipage}
	\begin{minipage}[c]{0.4\textwidth}
		\begin{tikzpicture}
			\begin{loglogaxis}[log basis x=2, log basis y=2, xlabel=Block size / words, ylabel=Blocks per set, zlabel=execution time / ns,
					xtick={1,2,4,8,16,32,64},
					ytick={1,2,4},
					view = {120}{35},% important to draw x,y in increasing order
					xmin = 0,
					ymin = 0,
					xmax = 128,
					ymax = 8,
					zmin = 0,
					unbounded coords = jump,
					colormap={pos}{color(0cm)=(white); color(6cm)=(orange)}
				]
				\addplot3[surf,mark=none, shader=flat, draw=mapped color!80!black] file {source/python/quicksort.dat};

			\end{loglogaxis}
		\end{tikzpicture}
	\end{minipage}
	\caption{Test results for quicksort with a cache size of 512 words.}
	\label{fig:quickResults}
\end{figure}
At first glance these results seem very similar to those obtained for the bubblesort algorithm except for the fact that quicksort is 
two orders of magnitude faster. Even though there is a significant performance gain when increasing block size, it can be seen that 
quicksort has less relative performance gain from increased block size than bubblesort, as
predicted. The large performance gain is the result of having a higher probability of not incurring a cache miss when swapping records
when the block size is appropriately large. 
Also, the performance gain from increased associativity is neglible except for block sizes of 32 words or larger.
