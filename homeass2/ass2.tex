\section{Assigment 2}
\subsection{Question 2.1}
The pipeline hazard associated with branches is called a control hazard.
\subsection{Question 2.2}
When the processor determines whether to branch or not in the second stage of the pipeline, a single instruction will always be
executed after the branch instruction.
\subsection{Question 2.3}
If the machine only has a single memory for both instructions and data, any instruction that tries to access memory will prevent
the fetch stage from being able to fetch the next instruction. Thus causing a structural hazard.

\subsection{Question 2.4}
\lstinputlisting[caption=code with use after load hazards marked., label=2-4]{source/ass2-4.s}
Since an instruction that uses a loaded value must be executed at least 1 clockcycle after the load instruction, the code has
2 use after load hazards as seen in the listing above.

\subsection{Question 2.5}
Inserting a single nop instruction after every use after load instruction in listing \ref{2-4} increases the number of instructions
in the loop body (excluding the beq and last nop instructions) to 10.
\subsection{Question 2.6}
The loop will repeat if the value in register \$t5 is zero. \$t5 is set in the previous instruction if the value in \$t4 is less
than the value in \$s0. Since this is always true \$t5 will be set to 1 on the first pass through the loop. However, since the
branch is determined in the ID stage, \$t5 has not yet bet set and a data hazard occurs, causing the branch to be taken in the
first pass through the loop. Thus the loop executes twice, bringing the total number of instructions (including nops to avoid
load-use data hazards) is 27.
\subsection{Question 2.7}
By moving all the load instructions to the top of the loop and placing any one of the addu instructions in the branch delay
slot removes the need for nop operations. The modified code is seen in the listing below
\lstinputlisting[caption=modified code without nops., label=2-7]{source/ass2-7.s}
Removing the 3 nop operations in the loop reduces the number of instructions executed to 21
