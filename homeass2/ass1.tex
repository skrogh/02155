\section{Assigment 1}
\subsection{Question 1.1}
\begin{description}
\item[ADD]
	\begin{description}
	\item[IF]
	 Instruction is fetched and put into IF/ID registers.
	 PC+4 is put into IF registers and - since no branching operations were before this one is loaded baxck into the PC. The path for the PC is thus not important any more.
	\item[ID]
	 0xBADDECAF is loaded from \$09 (09) and stored in ID/EX registe.rs.
	 0xB01DFACE is loaded from \$10 (0a) and stored in ID/EX regs
	 Last 4 bytes of instruction is signextended.
	 Writeback register adress \$08 (08) is passed on to ID/EX regs (though not
	 shown in MIPSPipe).
	\item[EX]
	 0xBADDECAF and 0xB01DFACE is added giving 0x6AFBE77D (sorry not a word)
	 Writeback register adress \$08 (08) is passed on to EX/MEM regs (though not
	 shown in MIPSPipe).
	\item[MEM]
	 Nothing stored to memory.
	 0x6AFBE77D passed on to MEM/WB regs.
	 Writeback register adress \$08 (08) is passed on to MEM/WB regs (though not
	 shown in MIPSPipe)
	\item[WB]
	 0x6AFBE77D is written into the register \$08 (08). 
	\end{description}	
\item[LW]
	\begin{description}
	\item[IF]
	 Instruction is fetched and put into IF/ID registers.
	 PC+4 is put into IF registers and - since no branching operations were before this one is loaded baxck into the PC. The path 	for the PC is thus not important any more.
	\item[ID]
	 The address 0x8001fff0 is loaded from \$09 (09) and stored in ID/EX regs.
	 The offset (0) is signextended and passed on to the ID/EX regs.
	 Writeback register adress \$08 (08) is passed on to ID/EX regs (though not
	 shown in MIPSPipe).
	\item[EX]
	 Offset and address is added and the final address is stored in the EX/MEM regs.
	 Writeback register adress \$08 (08) is passed on to EX /MEM regs (though not
	 shown in MIPSPipe).
	\item[MEM]
	 0x0BA11E70 is loaded from ram address 0x8001fff0 and stored in MEM/WB
	 Writeback register adress \$08 (08) is passed on to MEM/WB regs (though not
	 shown in MIPSPipe).
	\item[WB]
	 0BA11E70 is written to reg \$08
	\end{description}
\item[SW]
	\begin{description}
	\item[IF]
	 Instruction is fetched and put into IF/ID registers.
	 PC+4 is put into IF registers and - since no branching operations were before this one is loaded baxck into the PC. The path for the PC is thus not important any more.
	\item[ID]
	 The address 0x8001fff0 is loaded from \$09 (09) and stored in ID/EX regs.
	 The data 0x0DEBA7E is loaded from \$08 (08) and stored in ID/EX regs.
	 The offset (4) is signextended and passed on to the ID/EX regs.
	\item[EX]
	 Offset and address is added and the final address is stored in the EX/MEM regs.
	 Data is passed on to EX/MEM regs.
	\item[MEM]
	 0x0DEBA7E0 is  stored to ram address 8001fff4
	\item[WB]
	 No write back.
	\end{description}
\item[BEQ]
	\begin{description}
	\item[IF]
	 Instruction is fetched and put into IF/ID registers.
	 PC+4 is put into IF/ID registers.
	\item[ID]
	 The values in \$01 and \$02, both 0x00000000 are loaded into the ID/EX regs.
	 The relative jump (5) is sign extended and loaded into the ID/EX regs.
	 PC+4 (PC at call time) is passed on to ID/EX regs.
	\item[EX]
	 Relative jump is multiplied by 4 and added to PC+4 (PC at call time) then stored in EX/MEM regs.
	 0x00000000 and 0x00000000 are compared, as they are equal a 1 is passed to the EX/MEM regs. 
	\item[MEM]
	 PC is set to PC at calltime + 4 + relative jump*4.
	\item[WB]
	 no write back.
	\end{description}
\end{description}

\subsection{Question 1.2}
 For operations with write back 5, for SW that value is in the ram after 4 cycles.

\subsection{Question 1.3}
 ADD does not use the memory stage (BEQ doesn't use the ram, but does something here)
 SW and BEQ does not use the WB stage.
