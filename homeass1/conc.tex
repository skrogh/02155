\section{Conclusion}

\subsection{What is polling?}
Busy-waiting polling is waiting for a bit to be set, reset or first one, then
the other.

\subsection{How does the computer know whether a number is signed or unsigned?}
The computer does not differentiate between signed and unsinged numbers. This is
the result of using 2's complement for negative numbers. Addision and
subtraction can be peformed with the same algorithms for signed and unsigned
numbers. Thus there is no need for the computer to differentiate between signed
and unsigned numbers. It is up the programmer to keep track of number representation.
Some languages makes this easier, by doing some of the work for the programmer - or warning him, if he might be mixing representations.

\subsection{What is an overflow?}
When an operation is performed on one or more numbers and the result exeeds that
of which is possible to contain within the given representation of the resulting
number an overflow can (in special cases eg. with saturating math this is not
the case) result in an overflow, where the result - so to speak - \emph{loops
back}.

\subsection{Why are there non printable characters?}
Non printable characters function as control signals for other programs,terminals or printers.
The control signals sent to a terminal typically involve changing the position of the
cursor. Control characters intended for other programs can, for example, be used to 
acknowledge that a piece of information has been recieved.
