\section{Assignment 10 }
If the CPU keeps executing instructions after the last line of code has been executed it simply
executes whatever lies after the program in memory. If the program is uploaded to the mips simulator
nothing happens as the memory is filled with nop operations. 

If the .data directives are removed from the code, it still executes as intended and produces the sum
of 5 and 7 in register \$s0. This is only because the instructions with binary codes 5 and 7 do nothing.
If the ".word 7" segment in the code was changed to ".word 8", the program would jump to the address 0x00!
Thus we can, as expected, see that the cpu executes whatever instruction the PC is pointing to. And that the
.data directive causes the assembler to place the following code segment in an area of memory which the PC
should not point to under normal operation.
