\section{Assignment 3}
We will investigate the nature of simple arithmetics on a 32-bit \emph{MIPS}
processor. The operations we will look at are: \emph{addu, subu, add, sub}.

\subsection{\emph{addu}, \emph{subu}}
Starting with \emph{addu} and \emph{subu} we perform the following operations
\ref{optable}

 \begin{table}[ht]
\centering
\begin{tabular}{c c c c c c c c}
first arg. & as unsigned & as signed &  & second arg. / result & as unsigned &
as signed\\\hline
0x00000001 & 1 & 1 & + & 0x00000001 & 1 & 1\\
&  &  & = & 0x00000002 & 2 & 2\\ \hline
0x00000001 & 1 & 1 & + & 0x7FFFFFFF & 2147483647 & 2147483647\\
&  &  & = & 0x80000000 & 2147483648 & -2147483648\\\hline
0x00000001 & 1 & 1 & + & 0xFFFFFFFE & 4294967295 & -1\\
&  &  & = & 0x00000000 & 0 & 0\\\hline
0x00000002 & 2 & 2 & - & 0x00000001 & 1 & 1\\
&  &  & = & 0x00000001 & 1 & 1\\\hline
0x00000000 & 0 & 0 & - & 0x00000001 & 1 & 1\\
&  &  & = & 0xFFFFFFFF & 4294967295 & -1\\\hline
0x80000000 & 2147483648 & -2147483648 & - & 0x00000001 & 1 & 1\\
&  &  & = & 0x7FFFFFFF & 2147483647 & 2147483647\\\hline
\end{tabular}
\caption{Table of some 32-bit arithmetic operations}
\label{optable}
\end{table}
We note that overflow occurs differently for diffrerent representations of both
input and output - it depends on the programmers choice of representation, the
computer doesn't care - .
The source is given in listing \ref{lst:ass3} \lstinputlisting[caption={Source for assignment 3},
label={lst:ass3}]{source/ass3.s}

\subsection{\emph{add}, \emph{sub}}
These operations are supposed to generate a warning, a so-called \emph{trap} on
overflow. However I (we) can not figure out how this is expressed in the
\emph{MIPS}-simulator. An interrupt is supposed to be generated, saving the
processor state and \emph{PC} for backtracing the problem and possibly
recovering.
